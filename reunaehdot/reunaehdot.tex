\documentclass[a4paper, 12pt, finnish]{report}
\usepackage[utf8]{inputenc}
\usepackage{amsfonts}
\usepackage[finnish]{babel} %suomenkieli
\usepackage{titlesec}
\titleformat{\chapter}
{\Large\bfseries} % format
{}                % label
{0pt}             % sep
{\huge}           % before-code

\usepackage{hyperref}
\hypersetup{pdfpagemode=UseNone, pdfstartview=FitH, colorlinks=true,urlcolor=red,linkcolor=blue,citecolor=black,pdftitle={Opiskelijakeskuksen reunaehdot},pdfauthor={AYY}}
\setlength{\parindent}{0mm}
\setlength{\emergencystretch}{15pt}
\newcommand*{\findate}{\the\day.\the\month.\the\year}

\begin{document}


\chapter{Pohjaesitys Opiskelijakeskuksen reunaehdoista}
Edustajisto 5/2017\\
Hyväksytty ylioppilaskunnan hallituksen kokouksessa 12.5.2017\\

Tässä dokumentissa määritellään Aalto-yliopiston ylioppilaskunnan (AYY), Aalto-yliopiston kauppatieteiden ylioppilaat ry:n (KY) ja Teknologföreningenin (TF) yhteisen rakennusprojektin, Opiskelijakeskuksen, reunaehdot ja tahtotila sekä kirjataan Osapuolien aikeet projektin suhteen.
Opiskelijakeskuksen Osapuolilla tarkoitetaan AYY:aa, KY:ta ja TF:ä.


\subsection*{Työorganisaatio}
\textit{Työryhmä.} Hankkeen käytännön edistämisestä vastaava ryhmä, johon kuuluu kustakin Osapuolesta enintään kolme Osapuolen valitsemaa edustajaa. Osapuolien edustajilla Työryhmässä tulee olla mandaatti hankkeen käytännön edistämiseksi.
Työryhmä viestii keskenään avoimesti. Työryhmä viestii ulkoisesti yhdellä äänellä.
Ohjausryhmä. Hankkeen valmistelun tuloksia vietäessä eteenpäin päättäviin elimiin viedään asiat ohjausryhmälle, jossa paikalla tai edustettuina ovat Osapuolten toimeenpanevien elimien puheenjohtajat. Ohjausryhmän tehtävänä on mm. sopia yhteisistä päätösesityksistä.


\end{document}
