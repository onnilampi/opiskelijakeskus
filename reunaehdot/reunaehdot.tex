\documentclass[a4paper, 12pt, finnish]{report}
\usepackage[utf8]{inputenc}
\usepackage{amsfonts}
\usepackage[finnish]{babel} %suomenkieli
\usepackage{titlesec}
\usepackage{enumerate}
\usepackage{eurosym}
\titleformat{\chapter}
{\Large\bfseries} % format
{}                % label
{0pt}             % sep
{\huge}           % before-code

\usepackage{hyperref}
\hypersetup{pdfpagemode=UseNone, pdfstartview=FitH, colorlinks=true,urlcolor=red,linkcolor=blue,citecolor=black,pdftitle={Opiskelijakeskuksen reunaehdot},pdfauthor={AYY}}
\setlength{\parindent}{0mm}
\setlength{\emergencystretch}{15pt}
\newcommand*{\findate}{\the\day.\the\month.\the\year}

\begin{document}


\chapter{Pohjaesitys Opiskelijakeskuksen reunaehdoista}
Edustajisto 6/2017\\
Hyväksytty ylioppilaskunnan hallituksen kokouksessa 20.6.2017\\

Tässä dokumentissa määritellään Aalto-yliopiston ylioppilaskunnan (AYY), Aalto-yliopiston kauppatieteiden ylioppilaat ry:n (KY) ja Teknologföreningenin (TF) yhteisen rakennusprojektin, Opiskelijakeskuksen, reunaehdot ja tahtotila sekä kirjataan Osapuolien aikeet projektin suhteen.
Opiskelijakeskuksen Osapuolilla tarkoitetaan AYY:aa, KY:ta ja TF:ä.


\section*{Työorganisaatio}
\textit{Työryhmä.} Hankkeen käytännön edistämisestä vastaava ryhmä, johon kuuluu kustakin Osapuolesta enintään kolme Osapuolen valitsemaa edustajaa.
Osapuolien edustajilla Työryhmässä tulee olla mandaatti hankkeen käytännön edistämiseksi.\\

Työryhmä viestii keskenään avoimesti. Työryhmä viestii ulkoisesti yhdellä äänellä.\\

\textit{Ohjausryhmä.} Hankkeen valmistelun tuloksia vietäessä eteenpäin päättäviin elimiin viedään asiat ohjausryhmälle, jossa paikalla tai edustettuina ovat Osapuolten toimeenpanevien elimien puheenjohtajat.
Ohjausryhmän tehtävänä on mm. sopia yhteisistä päätösesityksistä.

\section*{Opiskelijakeskus ja visio}
Opiskelijakeskus on opiskelijoiden kohtaamis- ja oleskelupaikka, johon jokaisella Aaltolaisella on syy tulla.
Opiskelijakeskuksessa on opiskelijatoimintaa ja tätä tukevaa liike- ja juhlatoimintaa, sekä opiskelijoiden harjoittamaa muuta liiketoimintaa.
Opiskelijoille suunnatun toiminnan ja suunnattujen palveluiden tulee olla Opiskelijakeskuksen tärkein tavoite, niin kauan kun sen rakentavat Osapuolet eivät yksimielisesti päätä tämän muuttamisesta.\\

Opiskelijakeskus tulee olemaan pitkäaikainen tila TF:n, KY:n ja AYY:n jäsenille ja laajemmin Aalto-yliopiston opiskelijoille.
Tämä tulee huomioida rakennussunnittelussa: suunnittelussa tulee pyrkiä modulaarisiin ja muokattaviin ratkaisuihin, jotka muokkautuvat ja muuntuvat tarvittaessa siten, että ne palvelevat myös Osapuolten tulevaisuuden tarpeita, jotka voivat olla nykypäivään verrattuna muuttuneita tai uusia.\\

Opiskelijakeskuksen yleiset tilat muodostavat kokonaisvaltaisen, yhtenäisen ja monipuolisen palvelukokonaisuuden.


\section*{Opiskelijakeskuksesta syntyvien kustannusten jakaminen}
Ennen rakennussuunnittelua syntyvät kustannukset jaetaan tasan Osapuolten kesken. Myöhempien vaiheiden kustannuksista päätetään erikseen rahoitus- ja rakennuspäätösten yhteydessä.\\

Ennen rakennussuunnittelua syntyvät oletetut kokonaiskustannukset ovat seuraavat:

\begin{enumerate}
    \item Tilaohjelman päivitys tai laatiminen: max 5000 \euro{}.
    \item Arkkitehd(e)ille lähetettävän tarjouspyynnön teko: max 5000 \euro{}.
\end{enumerate}

Näiden kustannusten hyväksymiseen on Työryhmällä mandaatti.
Näihin lukeutumattomien odottamattomien yhteisten, ennen rakennussunnittelua syntyvien kustannusten aiheuttamiseksi tulee olla kaikkien Osapuolten päätös.\\

Myöhempien vaiheiden kustannusten jakoperusteesta päätetään samalla, kun päätetään kustannusten aiheuttamisesta.
Jakoperusteena voidaan käyttää esimerkiksi Osapuolten suunniteltuja osuuksia Opiskelijakeskushankkeen kokonaiskerrosalasta tai muuta kustannusten aiheutumisen mukaista jakoa.


\section*{Varainhankinta}
Kaikkien Osapuolten tulee, muista Osapuolista riippumatta, voida harjoittaa varainhankintaa rakennuskustannustensa kattamista varten.
Osapuolet viestivät varainhankinnasta keskenään avoimesti.


\section*{Yhtiömuoto}
Opiskelijakeskuksen omistussuhteen ja yhtiömuodon tulee mahdollistaa tässä dokumentissa määriteltyjen reunaehtojen toteutuminen.
Sen tulee myös mahdollistaa tilojen käyttötarkoituksen ja ulkomuodon muuttaminen.
Opiskelijakeskuksen kiinteistökokonaisuutta hallitseva yhtiö tulee määritellä voittoa tavoittelemattomaksi.
Lunastuslausekkeet tulee laatia siten, että mahdollisten ulkopuolisten rahoittajien vaatimukset Opiskelijakeskuksen tiloille ja niiden käytölle toteutuvat.\\

Opiskelijakeskuksen kiinteistökokonaisuutta hallitseva yhtiön toimintaperiaatteista sovitaan ennen rakennussuunnittelun aloittamista.


\section*{Yhteiskäyttötilat}
Osapuolet pyrkivät löytämään synergioita Opiskelijakeskuksen päällekkäisten toimintojen ja tilojen yhdistämisestä.
Tämä voidaan mahdollistaa esimerkiksi yhteisessä käytössä olevien tilojen muodossa.
Yleisessä ja/tai yhteisessä käytössä olevien tilojen suunnittelu ja omistus tulee toteuttaa siten, että TF:n anniskelu näissä mahdollistetaan kulloinkin voimassa olevan alkoholilainsäädännön puitteissa.\\

Yleisessä käytössä olevien tilojen omistuksesta, hallinnoinnista ja käytöstä sovitaan ennen rakennussuunnittelun aloittamista.

\section*{Yrityssuhteet}
Opiskelijakeskuksessa tapahtuvan yrityssuhdetoiminnan periaatteista sovitaan erikseen Osapuolten kesken ennen hankkeen toteutumista.


\section*{Julkisivu ja piha}
Opiskelijakeskuksen julkisivun (seinät ja katto) ulkonäöstä ja käytöstä tulee Osapuolien yhdessä voida päättää muista Osapuolten ulkopuolisista osapuolista riippumatta.\\

Opiskelijakeskuksen ulkopuolella tulee olla pihatilaa ja mahdollisesti terassi.
Mahdollisesta terassista tulee voida selkeästi rajata anniskelualue TF:n opiskelijaravintolan tarpeisiin sekä TF:n anniskelutoimintaa varten kulloinkin voimassaolevan alkoholilainsäädännön mukaisesti.\\

Osapuolten toiminnasta aiheutuva Opiskelijakeskuksen sisäinen ja ulos suuntautuva melu tulee sallia, mutta rakennussuunnittelussa otetaan huomioon melu ja minimoida tämän häiritsevyys (sekä sisä- että ulkopuolelle).


\section*{Omat tilat}
Osapuolilla tulee olla vapaus hallita ja tehdä vapaasti muutoksia itse omistamissaan tiloissaan, ilman muiden osapuolien lupaa.
Resurssi-intensiiviset tilat, esimerkiksi keittiöt ja märkätilat pyritään toteuttamaan toistensa läheisyyteen mahdollisuuksien mukaan käyttötarpeet huomioiden.


\section*{Tontti}
Opiskelijakeskuksen tontin tulee mahdollistaa muiden reunaehtojen (kuten pihatilan ja terassin) toteutuminen.\\

Tontin tulee ensisijaisesti olla Osapuolten yhteisomistuksessa tai Osapuolten yhteisesti omistaman yhtiön omistuksessa.\\

Vaihtoehtoisesti Opiskelijakeskus voidaan rakentaa Osapuolten ulkopuoliselta osapuolelta vuokratulle tontille, mutta tässä tapauksessa tontin vuokrasopimuksen tulee olla erittäin pitkä (esimerkiksi 50-100 vuotta).
Tässä tapauksessa tämä tontin vuokraava osapuoli ei saa olla yhtiö, jossa yksi Osapuoli on enemmistöosakas, tai muu yksikkö, jossa yksi Osapuoli on määrävässä asemassa.


\newpage
\section*{AYY:n reunaehdot}
\begin{enumerate}
    \item{Opiskelijakeskus toimii yhteisön rakentajana. Tilojen muunneltavuus ja monikäyttöisyys on keskiössä koko projektissa.}
        \begin{enumerate}[I.]
            \item{Jokaisella AYY:n jäsenellä on perusteltu syy käyttää Opiskelijakeskuksen palveluita, esimerkiksi seuraavilla tavoilla mutta ei näihin rajoittuen:}
                \begin{itemize}
                    \item{TF:n ruokalassa asiointi}
                    \item{Oleskelutilan käyttö}
                    \item{Juhlatiloissa järjestettävään tapahtumaan osallistuminen}
                    \item{Vapaaehtoistilojen käyttö}
                    \item{Monipuolinen palvelupiste}
                    \item{Julkisen tilan hyödyntäminen tapahtumiin}
                \end{itemize}

            \item{Jäsenistön käytössä on monipuolista ja muokattavaa palvelutilaa, jonka käyttöoikeus voi olla vaikkapa yhdistystuen muoto.}
        \end{enumerate}

    \item{AYY:n keskustoimisto ja kaikki nykyiset palvelutoiminnot siirtyvät uuteen Opiskelijakeskukseen kustannustehokkaaksi suunniteltuun toimitilaan.}
        \begin{enumerate}[I.]
            \item{AYY:n nykyiselle keskustoimisto-omistukselle on saatava uutta ja järkevää käyttöä AYY:n piiristä tai sen ulkopuolelta.}
        \end{enumerate}

    \item{Uudessa Opiskelijakeskuksessa vanhan keskustoimiston palvelutaso paranee.}
        \begin{enumerate}[I.]
            \item{Tiloja suunnitellaan siten, että ne tuovat merkittävää lisäarvoa nykyisille toiminnoille, mutta ovat silti muunneltavissa tulevaisuuden tarpeisiin.}
            \item{Erityisesti palvelupisteiden toiminnallisuus ja muunneltavuus on otettava tilan tehokkuuden kannalta vakavasti.}
        \end{enumerate}
    \item{AYY:n vapaaehtoisille suunnitellaan Opiskelijakeskukseen laadukkaat ja toimivat työtilat, sekä tilaa vapaa-ajan viettoon. Osa tai kaikki tästä tilasta voi olla yhteisissä tiloissa muiden toimijoiden kanssa tai yhteydessä niihin, mutta olennaista on että ne palvelevat koko AYY:n vapaaehtoiskenttää.}
        \begin{enumerate}[I.]
            \item{Tiloihin tulee voida mahdollistaa pääsy ympäri vuorokauden.}
            \item{Tilat toimivat myös normaalin jäsenistön projektiluontoisessa käytössä.}
        \end{enumerate}

    \item{Mikäli Opiskelijakeskuksen välittömään läheisyyteen on mahdollista rakentaa opiskelija-asuntoja, ensisijaisena rakennuttajana toimii AYY.}
\end{enumerate}


\newpage
\section*{KY:n reunaehdot}
\begin{enumerate}
    \item{Kauppatieteiden ylioppilaiden uusien tilojen Otaniemessä tulee tukea kauppatieteilijöiden edellytyksiä, kuitenkin ottaen huomioon Aalto-yliopiston kaikki opiskelijat.}

    \item{KY:llä tulee olla jäsentilat ja tilat KY:n yhdistyksille, ainejärjestöille sekä jaostoille. Opiskelijakeskuksessa tulee olla yhteistä tilaa koko Aalto-yhteisön vapaaehtoisille.}

    \item{KY:n toiminnan nykyinen palvelutaso on säilytettävä, tai sen on parannuttava.}

    \item{KY:llä tulee olla omat tilat, joita KY:n tulee itse voida hallinnoida ja joista sen tulee voida päättää täysin itse, ilman muiden osapuolien lupaa. Nämä tilat koostuvat eri osista:}
        \begin{enumerate}[I.]
            \item{\textit{Jäsentilat.} KY:llä tulee olla sen jäsenille tilaa, johon kuuluu ainakin} 
                \begin{itemize}
                    \item{Saunatilat}
                    \item{Juhlatilat, joihin kuuluvat myös}
                        \begin{itemize}
                            \item{Keittiö juhlien toteuttamiseksi (mahdollisesti osa isompaa kokonaisuutta)}
                            \item{Näyttämö sekä takatilat}
                        \end{itemize}
                    \item{Kokoustilat}
                    \item{Kellareissa varastotilaa ja yhdistysten tilaa (talotekniikan tulee mahdollistaa yhdistysten tilojen ylläpidon kellarikerroksessa) }
                    \item{Oleskelutilaa}
                    \item{Toimistotilat}
                    \item{Jäsenten omien yhdistysten tilat}
                \end{itemize}

            \item{\textit{Yhteiset tilat.}}
                \begin{itemize}
                    \item{Verstas (voi olla yhteisomistuksessa ja/tai yhteisesti hallinnoitu muun tai muiden Osapuolien kanssa, tai jonkun muun Osapuolen hallinnoima)}
                    \item{Harjoitustilaa orkestereille/bändeille (voi olla yhteisomistuksessa ja/tai yhteisesti hallinnoitu muun tai muiden Osapuolien kanssa, tai jonkun muun Osapuolen hallinnoima)}
                    \item{Studio (valokuvaus/videokuvausmahdollisuus, voi olla yhteisomistuksessa ja/tai yhteisesti hallinnoitu muun tai muiden Osapuolien kanssa, tai jonkun muun Osapuolen hallinnoima)}
                \end{itemize}
        \end{enumerate}
    \item{Kaikkiin edellä mainittuihin tiloihin on voitava mahdollistaa pääsy 24/7 jäsenille.}
\end{enumerate}


\newpage
\section*{TF:n reunaehdot}
\begin{enumerate}
    \item{TF:n tulee voida toimia kaikille avoimena kohtaamispaikkana Opiskelijakeskuksessa tarjoamalla tiloja ja palveluita, joita kaikki Aaltolaiset haluavat käyttää. TF:n nykyisen statuksen ja identiteetin tulee säilyä tai parantua. Lisäksi palvelutason jäsenille ja jäsenetujen tulee säilyä tai parantua.}
    \item{TF:llä tulee olla mahdollisuus harjoittaa liiketoimintaa Opiskelijakeskuksessa.}
    \item{TF:llä tulee olla yksinoikeus opiskelija- ja anniskeluravintolatoimintaan Opiskelijakeskuksessa. Opiskelijakeskuksen ja sitä mahdollisesti hallitsevan yhtiön säännöt tulee määritellä sekä Opiskelijakeskuksen suunnittelu, omistus ja hallinta toteuttaa niin, että TF:llä on edellytykset opiskelijaravintola-, ravintola- ja juhlatilatoiminnan (sis. anniskelu) harjoittamiseksi Opiskelijakeskuksessa. Opiskelijakeskuksen muiden tilojen ei tule kilpailla TF:n harjoittaman ravintola-, tilausravintola- ja anniskelutoiminnan kanssa esimerkiksi tarjoamalla vastaavia palveluita.}
    \item{TF:llä tulee olla omat tilat, jotka muodostavat yhtenäisen kokonaisuuden. Näitä tiloja TF:n tulee itse voida hallinnoida ja näistä sen tulee voida päättää täysin itse, ilman muiden osapuolien lupaa. Kaikkiin tiloihin, joista TF omistaa osan tai jotka se omistaa kokonaan on voitava mahdollistaa ympärivuorokautinen pääsy TF:n jäsenille. TF:n tilat koostuvat eri osista:}
        \begin{enumerate}[I.]
            \item{\textit{Kaikille avoimet tilat.} TF:n tulee voida toimia kaikille avoimena kohtaamispaikkana ja TF:n tulee voida harjoittaa opiskelijaravintola- ja anniskeluravintolatoimintaa. TF:llä tulee olla tätä tarkoitusta varten ainakin ravintolan ruokasali ja yleisen käytön mahdollistava juhlasali TF:n omiin ja vuokratarpeisiin. Ravintolan ruokasali voi myös toimia juhlasalina.}
            \item{\textit{Jäsentilat.} TF:llä tulee olla sen jäsenille tarkoitettua tilaa, johon kuuluvat ainakin}
                \begin{itemize}
                    \item{Oleskelutilaa}
                    \item{Juhlatilaa}
                    \item{Sauna ja siihen kuuluvat tilat}
                    \item{Tilaa TF:n jäsenien yhdistyksille ja tapahtumille}
                        \begin{itemize}
                            \item{Varastotilaa}
                            \item{Orkesterin harjoitustila tulee mahdollistaa Opiskelijakeskuksessa (voi olla yhteisomistuksessa ja/tai yhteisesti hallinnoitu muun tai muiden Osapuolien kanssa, tai jonkun muun Osapuolen hallinnoima)}
                        \end{itemize}
                    \item{Kokoustilaa}
                    \item{Toimistotilaa hallitukselle, kanslialle ja TF:n liiketoiminnalle}
                    \item{Keittiö jäsenien käyttöön}
                    \item{Verstas (ellei sitä järjestetä muualle) (voi olla yhteisomistuksessa ja/tai yhteisesti hallinnoitu muun tai muiden Osapuolien kanssa, tai jonkun muun Osapuolen hallinnoima)}
                \end{itemize}
            \item{\textit{Ravintolan ja muun liiketoiminnan tilat.} TF:n ravintolan ja muun liiketoiminnan (mukaan lukien anniskeluravintolan) toiminta tulee mahdollistaa vähintään seuraavien tilojen muodossa:}
                \begin{itemize}
                    \item{Ravintolan ruokasali sekä keittiö ja siihen kuuluvat tilat}
                    \item{Juhlasalin toimintaan liittyvät tekniset tilat (valot, ääni, yms.)}
                    \item{Lastauslaituri}
                    \item{TF:n ravintolan, kanslian ja liiketoiminnan henkilökunnan parkkipaikat tulee järjestää Opiskelijakeskuksen läheisyyteen}
                    \item{Säilytys- ja toimistotilaa TF:n yrityksille}
                \end{itemize}
        \end{enumerate}
\end{enumerate}


\newpage
\section*{Voimassaolo ja seuraavat askeleet}
Nämä reunaehdot tulevat voimaan tässä muodossa kun KY:n edustajisto, TF:n osakuntakokous sekä AYY:n edustajisto, eli Osapuolten korkeimmat päättävät elimet, ovat ne hyväksyneet tässä muodossa.
Tämän jälkeen näistä reunaehdoista ei poiketa ilman Osapuolten yksimielistä päätöstä.\\

Nämä reunaehdot ovat voimassa toistaiseksi.
Mikäli jokin Osapuoli jättäytyy hankkeesta, noudatetaan näitä reunaehtoja muiden Osapuolien välillä kunnes uudet reunaehdot on laadittu näiden jäljellä olevien Osapuolien toimesta.
Vetäytynyt taho on vastuussa kustannuksista jotka ovat aiheutuneet tai yhdessä kohdistettu heille ennen heidän vetäytymistä hankkeesta.\\

Rakennuspäätöksen tapahduttua ja kiinteistöyhtiötä muodostaessa asioista sovitaan tarkemmin, kuitenkin näitä reunaehtoja noudattaen, elleivät Osapuolet yksimielisesti toisin päätä.
Ennen hankkeen toteutumista Osapuolten on laadittava kaikki tarvittavat sopimukset, näitä reunaehtoja noudattaen.



\end{document}
